\chapter{INTRODUCCI\'ON}

\doublespacing
\addcontentsline{toc}{chapter}{INTRODUCCI\'ON}

En el presente informe Se explicara como es que se debe realizar un respaldo de la informacion, en este caso el respaldo de una base de datos en Oracle 11g Enterprise Edition para es uso del asistente grafico para copias de seguridad (Enterprise Manager).
Ademas se utilizara SQLDEVElOPER.exe para para conectar un usuario , tambien sirve para migracion de bases de datos de MySQL a Oracle.
Se explicara que tipos de backups se pueden realizar en Oracle, algunas recomendaciones de cuando realizar las copias de seguridad ademas de copias de seguridad en modo consola y de manera grafica

Una copia de los datos que se puede utilizar para restaurar y recuperar los datos se denomina copia de seguridad. Las copias de seguridad le permiten restaurar los datos después de un error. Con las copias de seguridad correctas, puede recuperarse de multitud de errores como:
●	Errores de medios.
●	Errores de usuario, por ejemplo, quitar una tabla por error.
●	Errores de hardware, por ejemplo, una unidad de disco dañada o la pérdida
●	permanente de un servidor.
●	Desastres naturales.

Además, las copias de seguridad de una base de datos son útiles para fines administrativos habituales, como copiar una base de datos de un servidor a otro, configurar la creación de reflejo de la base de datos y el archivo, etc.
Para impedir la perdida de datos se debe disponer de una estrategia de copia de seguridad, hacer copias de seguridad con regularidad, tambien se debe considerar los tipos de respaldo que soporta oracle, los respaldo y recuperacion y su procedimiento(el plan, que debe incluir, medios de soporte a utilizar, cuando realizarlo, periodicidad) y las herramientas(Dónde guardarlos - distancia y accesibilidad,Quienes realizan y manejan los  respaldos?, Verificación del respaldo, Registro, consejos para realizar los respaldos e instalaciones grandes).